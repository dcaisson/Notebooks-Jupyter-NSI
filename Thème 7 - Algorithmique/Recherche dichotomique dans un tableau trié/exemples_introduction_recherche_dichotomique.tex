\documentclass[usenames,dvipsnames]{beamer}

\usetheme{focus}

\usepackage[french]{babel}
\usepackage[T1]{fontenc}
\usepackage[utf8]{inputenc}

\usepackage{dsfont}

\usepackage{xcolor}
\usepackage{colortbl}

\usepackage{etex}

\usepackage{pstricks,pst-plot,pst-text,pst-tree,pst-eps,pst-fill,pst-node,pst-math}
\usepackage{pstricks-add,pst-xkey}

\usepackage{multicol}

\usepackage{tikz}

\newcommand*\circled[1]{\tikz[baseline=(char.base)]{
            \node[color=ForestGreen,shape=circle,draw,inner sep=2pt] (char) {#1};}}

\begin{document}

\title{La recherche dichotomique -- Exemples}

\date{}

\maketitle{}

\begin{frame}
  \frametitle{Premier exemple}
  \begin{itemize}
    \item On cherche la valeur 5 dans le tableau trié ci-dessous :

      \bigskip

      \renewcommand{\arraystretch}{1.4}
      \hspace{-7mm}\begin{tabular}{|*{9}{>{\centering}m{8mm}|}}
	\hline
	1 & 2 & 5 & 9 & 10 & 14 & 17 & 24 & 41\tabularnewline
	\hline
	\multicolumn{1}{c}{\color{red}$\uparrow$} & \multicolumn{1}{c}{} &  \multicolumn{1}{c}{} & \multicolumn{1}{c}{} & \multicolumn{1}{c}{} & \multicolumn{1}{c}{} & \multicolumn{1}{c}{} & \multicolumn{1}{c}{} & \multicolumn{1}{c}{\color{red}$\uparrow$}\tabularnewline
	\multicolumn{1}{c}{\color{red}déb.} & \multicolumn{1}{c}{} &  \multicolumn{1}{c}{} & \multicolumn{1}{c}{} & \multicolumn{1}{c}{} & \multicolumn{1}{c}{} & \multicolumn{1}{c}{} & \multicolumn{1}{c}{} & \multicolumn{1}{c}{\color{red}fin}\tabularnewline
      \end{tabular}\pause{}
    \item La valeur du milieu est $10$ :

      \bigskip

      \renewcommand{\arraystretch}{1.4}
      \hspace{-7mm}\begin{tabular}{|*{9}{>{\centering}m{8mm}|}}
	\hline
	1 & 2 & 5 & 9 & \circled{10} & 14 & 17 & 24 & 41\tabularnewline
	\hline
	\multicolumn{1}{c}{\color{red}$\uparrow$} & \multicolumn{1}{c}{} &  \multicolumn{1}{c}{} & \multicolumn{1}{c}{} & \multicolumn{1}{c}{\color{ForestGreen}$\uparrow$} & \multicolumn{1}{c}{} & \multicolumn{1}{c}{} & \multicolumn{1}{c}{} & \multicolumn{1}{c}{\color{red}$\uparrow$}\tabularnewline
	\multicolumn{1}{c}{\color{red}déb.} & \multicolumn{1}{c}{} &  \multicolumn{1}{c}{} & \multicolumn{1}{c}{} & \multicolumn{1}{c}{\color{ForestGreen}mil.} & \multicolumn{1}{c}{} & \multicolumn{1}{c}{} & \multicolumn{1}{c}{} & \multicolumn{1}{c}{\color{red}fin}\tabularnewline
      \end{tabular}\pause{}
    \item Puisque $5<10$, on continue la recherche dans la moitié de gauche du tableau : pour cela, on change la valeur de {\color{red}fin}.\pause{}

     \smallskip

     \begin{center}
       {\color{red}fin}\ $\gets$\ {\color{ForestGreen}mil.} $-\ 1$
     \end{center}
  \end{itemize}
\end{frame}

\begin{frame}
 \begin{itemize}
   \item La recherche se poursuit dans la partie non grisée :

     \bigskip

      \renewcommand{\arraystretch}{1.4}
      \hspace{-7mm}\begin{tabular}{|*{9}{>{\centering}m{8mm}|}}
	\hline
	1 & 2 & 5 & 9 &\cellcolor[gray]{0.7} 10 &\cellcolor[gray]{0.7} 14 &\cellcolor[gray]{0.7} 17 &\cellcolor[gray]{0.7} 24 &\cellcolor[gray]{0.7} 41\tabularnewline
	\hline
	\multicolumn{1}{c}{\color{red}$\uparrow$} & \multicolumn{1}{c}{} &  \multicolumn{1}{c}{} & \multicolumn{1}{c}{\color{red}$\uparrow$} & \multicolumn{1}{c}{} & \multicolumn{1}{c}{} & \multicolumn{1}{c}{} & \multicolumn{1}{c}{} & \multicolumn{1}{c}{}\tabularnewline
	\multicolumn{1}{c}{\color{red}déb.} & \multicolumn{1}{c}{} &  \multicolumn{1}{c}{} & \multicolumn{1}{c}{\color{red}fin} & \multicolumn{1}{c}{} & \multicolumn{1}{c}{} & \multicolumn{1}{c}{} & \multicolumn{1}{c}{} & \multicolumn{1}{c}{}\tabularnewline
      \end{tabular}\pause{}
    \item La valeur du milieu est $2$ :

      \bigskip

      \renewcommand{\arraystretch}{1.4}
      \hspace{-7mm}\begin{tabular}{|*{9}{>{\centering}m{8mm}|}}
	\hline
	1 & \circled{2} & 5 & 9 &\cellcolor[gray]{0.7} 10 &\cellcolor[gray]{0.7} 14 &\cellcolor[gray]{0.7} 17 &\cellcolor[gray]{0.7} 24 &\cellcolor[gray]{0.7} 41\tabularnewline
	\hline
	\multicolumn{1}{c}{\color{red}$\uparrow$} & \multicolumn{1}{c}{\color{ForestGreen}$\uparrow$} &  \multicolumn{1}{c}{} & \multicolumn{1}{c}{\color{red}$\uparrow$} & \multicolumn{1}{c}{} & \multicolumn{1}{c}{} & \multicolumn{1}{c}{} & \multicolumn{1}{c}{} & \multicolumn{1}{c}{}\tabularnewline
	\multicolumn{1}{c}{\color{red}déb.} & \multicolumn{1}{c}{\color{ForestGreen}mil.} &  \multicolumn{1}{c}{} & \multicolumn{1}{c}{\color{red}fin} & \multicolumn{1}{c}{} & \multicolumn{1}{c}{} & \multicolumn{1}{c}{} & \multicolumn{1}{c}{} & \multicolumn{1}{c}{}\tabularnewline
      \end{tabular}\pause{}
    \item Puisque $5>2$, on continue la recherche dans la moitié de droite : pour cela, on change la valeur de {\color{red}déb.}\pause{}

    \smallskip

     \begin{center}
       {\color{red}déb.} $\gets$\ {\color{ForestGreen}mil.} $+\ 1$
     \end{center}
 \end{itemize} 
\end{frame}

\begin{frame}
  \begin{itemize}
    \item La recherche se poursuit dans la partie non grisée :

      \bigskip

      \renewcommand{\arraystretch}{1.4}
      \hspace{-7mm}\begin{tabular}{|*{9}{>{\centering}m{8mm}|}}
	\hline
	\cellcolor[gray]{0.7}1 &\cellcolor[gray]{0.7} 2 & 5 & 9 &\cellcolor[gray]{0.7} 10 &\cellcolor[gray]{0.7} 14 &\cellcolor[gray]{0.7} 17 &\cellcolor[gray]{0.7} 24 &\cellcolor[gray]{0.7} 41\tabularnewline
	\hline
	\multicolumn{1}{c}{} & \multicolumn{1}{c}{} &  \multicolumn{1}{c}{\color{red}$\uparrow$} & \multicolumn{1}{c}{\color{red}$\uparrow$} & \multicolumn{1}{c}{} & \multicolumn{1}{c}{} & \multicolumn{1}{c}{} & \multicolumn{1}{c}{} & \multicolumn{1}{c}{}\tabularnewline
	\multicolumn{1}{c}{} & \multicolumn{1}{c}{} &  \multicolumn{1}{c}{\color{red}déb.} & \multicolumn{1}{c}{\color{red}fin} & \multicolumn{1}{c}{} & \multicolumn{1}{c}{} & \multicolumn{1}{c}{} & \multicolumn{1}{c}{} & \multicolumn{1}{c}{}\tabularnewline
      \end{tabular}\pause{}
    \item La valeur du milieu est $5$ :

      \bigskip

      \renewcommand{\arraystretch}{1.4}
      \hspace{-7mm}\begin{tabular}{|*{9}{>{\centering}m{8mm}|}}
	\multicolumn{1}{c}{} & \multicolumn{1}{c}{} &  \multicolumn{1}{c}{\color{ForestGreen}mil.} & \multicolumn{1}{c}{} & \multicolumn{1}{c}{} & \multicolumn{1}{c}{} & \multicolumn{1}{c}{} & \multicolumn{1}{c}{} & \multicolumn{1}{c}{}\tabularnewline
	\multicolumn{1}{c}{} & \multicolumn{1}{c}{} &  \multicolumn{1}{c}{\color{ForestGreen}$\downarrow$} & \multicolumn{1}{c}{} & \multicolumn{1}{c}{} & \multicolumn{1}{c}{} & \multicolumn{1}{c}{} & \multicolumn{1}{c}{} & \multicolumn{1}{c}{}\tabularnewline
	\hline
	\cellcolor[gray]{0.7}1 &\cellcolor[gray]{0.7} 2 & \circled{5} & 9 &\cellcolor[gray]{0.7} 10 &\cellcolor[gray]{0.7} 14 &\cellcolor[gray]{0.7} 17 &\cellcolor[gray]{0.7} 24 &\cellcolor[gray]{0.7} 41\tabularnewline
	\hline
	\multicolumn{1}{c}{} & \multicolumn{1}{c}{} &  \multicolumn{1}{c}{\color{red}$\uparrow$} & \multicolumn{1}{c}{\color{red}$\uparrow$} & \multicolumn{1}{c}{} & \multicolumn{1}{c}{} & \multicolumn{1}{c}{} & \multicolumn{1}{c}{} & \multicolumn{1}{c}{}\tabularnewline
	\multicolumn{1}{c}{} & \multicolumn{1}{c}{} &  \multicolumn{1}{c}{\color{red}déb.} & \multicolumn{1}{c}{\color{red}fin} & \multicolumn{1}{c}{} & \multicolumn{1}{c}{} & \multicolumn{1}{c}{} & \multicolumn{1}{c}{} & \multicolumn{1}{c}{}\tabularnewline
      \end{tabular}\pause{}
    \item La recherche s'arrête car on a trouvé la valeur $5$.
    \end{itemize}
\end{frame}

\begin{frame}
  \frametitle{Second exemple}
  \begin{itemize}
    \item On cherche la valeur 15 dans le tableau trié ci-dessous :

      \bigskip

      \renewcommand{\arraystretch}{1.4}
      \hspace{-7mm}\begin{tabular}{|*{9}{>{\centering}m{8mm}|}}
	\hline
	1 & 2 & 5 & 9 & 10 & 14 & 17 & 24 & 41\tabularnewline
	\hline
	\multicolumn{1}{c}{\color{red}$\uparrow$} & \multicolumn{1}{c}{} &  \multicolumn{1}{c}{} & \multicolumn{1}{c}{} & \multicolumn{1}{c}{} & \multicolumn{1}{c}{} & \multicolumn{1}{c}{} & \multicolumn{1}{c}{} & \multicolumn{1}{c}{\color{red}$\uparrow$}\tabularnewline
	\multicolumn{1}{c}{\color{red}déb.} & \multicolumn{1}{c}{} &  \multicolumn{1}{c}{} & \multicolumn{1}{c}{} & \multicolumn{1}{c}{} & \multicolumn{1}{c}{} & \multicolumn{1}{c}{} & \multicolumn{1}{c}{} & \multicolumn{1}{c}{\color{red}fin}\tabularnewline
      \end{tabular}\pause{}
    \item La valeur du milieu est $10$ :

      \bigskip

      \renewcommand{\arraystretch}{1.4}
      \hspace{-7mm}\begin{tabular}{|*{9}{>{\centering}m{8mm}|}}
	\hline
	1 & 2 & 5 & 9 & \circled{10} & 14 & 17 & 24 & 41\tabularnewline
	\hline
	\multicolumn{1}{c}{\color{red}$\uparrow$} & \multicolumn{1}{c}{} &  \multicolumn{1}{c}{} & \multicolumn{1}{c}{} & \multicolumn{1}{c}{\color{ForestGreen}$\uparrow$} & \multicolumn{1}{c}{} & \multicolumn{1}{c}{} & \multicolumn{1}{c}{} & \multicolumn{1}{c}{\color{red}$\uparrow$}\tabularnewline
	\multicolumn{1}{c}{\color{red}déb.} & \multicolumn{1}{c}{} &  \multicolumn{1}{c}{} & \multicolumn{1}{c}{} & \multicolumn{1}{c}{\color{ForestGreen}mil.} & \multicolumn{1}{c}{} & \multicolumn{1}{c}{} & \multicolumn{1}{c}{} & \multicolumn{1}{c}{\color{red}fin}\tabularnewline
      \end{tabular}\pause{}
    \item Puisque $15>10$, on continue la recherche dans la moitié de droite du tableau : pour cela, on change la valeur de {\color{red}déb.}\pause{}

    \smallskip

     \begin{center}
       {\color{red}déb.} $\gets$\ {\color{ForestGreen}mil.} $+\ 1$
     \end{center}
  \end{itemize}
\end{frame}

\begin{frame}
 \begin{itemize}
   \item La recherche se poursuit dans la partie non grisée :

      \bigskip

      \renewcommand{\arraystretch}{1.4}
      \hspace{-7mm}\begin{tabular}{|*{9}{>{\centering}m{8mm}|}}
	\hline
	\cellcolor[gray]{0.7}1 &\cellcolor[gray]{0.7} 2 &\cellcolor[gray]{0.7} 5 &\cellcolor[gray]{0.7} 9 &\cellcolor[gray]{0.7} 10 & 14 & 17 & 24 & 41\tabularnewline
	\hline
	\multicolumn{1}{c}{} & \multicolumn{1}{c}{} &  \multicolumn{1}{c}{} & \multicolumn{1}{c}{} & \multicolumn{1}{c}{} & \multicolumn{1}{c}{\color{red}$\uparrow$} & \multicolumn{1}{c}{} & \multicolumn{1}{c}{} & \multicolumn{1}{c}{\color{red}$\uparrow$}\tabularnewline
	\multicolumn{1}{c}{} & \multicolumn{1}{c}{} &  \multicolumn{1}{c}{} & \multicolumn{1}{c}{} & \multicolumn{1}{c}{} & \multicolumn{1}{c}{\color{red}déb.} & \multicolumn{1}{c}{} & \multicolumn{1}{c}{} & \multicolumn{1}{c}{\color{red}fin}\tabularnewline
      \end{tabular}\pause{}
    \item La valeur du milieu est $17$ :

      \bigskip

      \renewcommand{\arraystretch}{1.4}
      \hspace{-7mm}\begin{tabular}{|*{9}{>{\centering}m{8mm}|}}
	\hline
	\cellcolor[gray]{0.7}1 &\cellcolor[gray]{0.7} 2 &\cellcolor[gray]{0.7} 5 &\cellcolor[gray]{0.7} 9 &\cellcolor[gray]{0.7} 10 & 14 & \circled{17} & 24 & 41\tabularnewline
	\hline
	\multicolumn{1}{c}{} & \multicolumn{1}{c}{} &  \multicolumn{1}{c}{} & \multicolumn{1}{c}{} & \multicolumn{1}{c}{} & \multicolumn{1}{c}{\color{red}$\uparrow$} & \multicolumn{1}{c}{\color{ForestGreen}$\uparrow$} & \multicolumn{1}{c}{} & \multicolumn{1}{c}{\color{red}$\uparrow$}\tabularnewline
	\multicolumn{1}{c}{} & \multicolumn{1}{c}{} &  \multicolumn{1}{c}{} & \multicolumn{1}{c}{} & \multicolumn{1}{c}{} & \multicolumn{1}{c}{\color{red}déb.} & \multicolumn{1}{c}{\color{ForestGreen}mil.} & \multicolumn{1}{c}{} & \multicolumn{1}{c}{\color{red}fin}\tabularnewline
      \end{tabular}\pause{}
    \item Puisque $15<17$, on continue la recherche dans la moitié de gauche : pour cela, on change la valeur de {\color{red}fin}.\pause{}

    \smallskip

     \begin{center}
       {\color{red}fin}\ $\gets$\ {\color{ForestGreen}mil.} $-\ 1$
     \end{center}
 \end{itemize} 
\end{frame}

\begin{frame}
  \begin{itemize}
    \item La recherche se poursuit dans la partie non grisée :

      \bigskip

      \renewcommand{\arraystretch}{1.4}
      \hspace{-7mm}\begin{tabular}{|*{9}{>{\centering}m{8mm}|}}
	\hline
	\cellcolor[gray]{0.7}1 &\cellcolor[gray]{0.7} 2 &\cellcolor[gray]{0.7} 5 &\cellcolor[gray]{0.7} 9 &\cellcolor[gray]{0.7} 10 & 14 &\cellcolor[gray]{0.7} 17 &\cellcolor[gray]{0.7} 24 &\cellcolor[gray]{0.7} 41\tabularnewline
	\hline
	\multicolumn{1}{c}{} & \multicolumn{1}{c}{} &  \multicolumn{1}{c}{} & \multicolumn{1}{c}{} & \multicolumn{1}{c}{} & \multicolumn{1}{c}{\color{red}$\uparrow$} & \multicolumn{1}{c}{} & \multicolumn{1}{c}{} & \multicolumn{1}{c}{}\tabularnewline
	\multicolumn{1}{c}{} & \multicolumn{1}{c}{} &  \multicolumn{1}{c}{} & \multicolumn{1}{c}{} & \multicolumn{3}{c}{\color{red}déb.\ =\ fin}  & \multicolumn{1}{c}{} & \multicolumn{1}{c}{}\tabularnewline
      \end{tabular}\pause{}
    \item La valeur du milieu est 14 :

      \bigskip

      \renewcommand{\arraystretch}{1.4}
      \hspace{-7mm}\begin{tabular}{|*{9}{>{\centering}m{8mm}|}}
	\multicolumn{1}{c}{} & \multicolumn{1}{c}{} &  \multicolumn{1}{c}{} & \multicolumn{1}{c}{} & \multicolumn{1}{c}{} & \multicolumn{1}{c}{\color{ForestGreen}mil.} & \multicolumn{1}{c}{} & \multicolumn{1}{c}{} & \multicolumn{1}{c}{}\tabularnewline
	\multicolumn{1}{c}{} & \multicolumn{1}{c}{} &  \multicolumn{1}{c}{} & \multicolumn{1}{c}{} & \multicolumn{1}{c}{} & \multicolumn{1}{c}{\color{ForestGreen}$\downarrow$} & \multicolumn{1}{c}{} & \multicolumn{1}{c}{} & \multicolumn{1}{c}{}\tabularnewline
	\hline
	\cellcolor[gray]{0.7}1 &\cellcolor[gray]{0.7} 2 &\cellcolor[gray]{0.7} 5 &\cellcolor[gray]{0.7} 9 &\cellcolor[gray]{0.7} 10 & \circled{14} &\cellcolor[gray]{0.7} 17 &\cellcolor[gray]{0.7} 24 &\cellcolor[gray]{0.7} 41\tabularnewline
	\hline
	\multicolumn{1}{c}{} & \multicolumn{1}{c}{} &  \multicolumn{1}{c}{} & \multicolumn{1}{c}{} & \multicolumn{1}{c}{} & \multicolumn{1}{c}{\color{red}$\uparrow$} & \multicolumn{1}{c}{} & \multicolumn{1}{c}{} & \multicolumn{1}{c}{}\tabularnewline
	\multicolumn{1}{c}{} & \multicolumn{1}{c}{} &  \multicolumn{1}{c}{} & \multicolumn{1}{c}{} & \multicolumn{3}{c}{\color{red}déb.\ =\ fin}  & \multicolumn{1}{c}{} & \multicolumn{1}{c}{}\tabularnewline
      \end{tabular}\pause{}
   \item Puisque $15>14$, on continue la recherche dans la moitié de droite du tableau : pour cela, on change la valeur de {\color{red}déb.}

    \smallskip

     \begin{center}
       {\color{red}déb.} $\gets$\ {\color{ForestGreen}mil.} $+\ 1$
     \end{center}
  \end{itemize} 
\end{frame}

\begin{frame}
 \begin{itemize}
   \item La partie non grisée est vide car {\color{red}déb.}\ $>$\ {\color{red}fin} : la recherche s'arrête. 

      \bigskip

      \renewcommand{\arraystretch}{1.4}
      \hspace{-7mm}\begin{tabular}{|*{9}{>{\centering}m{8mm}|}}
	\hline
	\cellcolor[gray]{0.7}1 &\cellcolor[gray]{0.7} 2 &\cellcolor[gray]{0.7} 5 &\cellcolor[gray]{0.7} 9 &\cellcolor[gray]{0.7} 10 &\cellcolor[gray]{0.7} 14 &\cellcolor[gray]{0.7} 17 &\cellcolor[gray]{0.7} 24 &\cellcolor[gray]{0.7} 41\tabularnewline
	\hline
	\multicolumn{1}{c}{} & \multicolumn{1}{c}{} &  \multicolumn{1}{c}{} & \multicolumn{1}{c}{} & \multicolumn{1}{c}{} & \multicolumn{1}{c}{\color{red}$\uparrow$} & \multicolumn{1}{c}{\color{red}$\uparrow$} & \multicolumn{1}{c}{} & \multicolumn{1}{c}{}\tabularnewline
	\multicolumn{1}{c}{} & \multicolumn{1}{c}{} &  \multicolumn{1}{c}{} & \multicolumn{1}{c}{} & \multicolumn{1}{c}{}  & \multicolumn{1}{c}{\color{red}fin} & \multicolumn{1}{c}{\color{red}déb.} & \multicolumn{1}{c}{} & \multicolumn{1}{c}{}\tabularnewline
      \end{tabular}\pause{}
    \item La valeur $15$ n'est donc pas présente dans le tableau initial.
 \end{itemize} 
\end{frame}

\end{document}

%%% Local Variables:
%%% mode: latex
%%% TeX-master: t
%%% End:
